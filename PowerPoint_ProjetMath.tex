\documentclass{beamer}

\usepackage[frenchb]{babel}
\usepackage[T1]{fontenc}
\usepackage[latin1]{inputenc}
\usepackage[]{amsmath}
\usepackage[]{amssymb}
 \setcounter{tocdepth}{1}

\usetheme{Warsaw}


\title[Présentation Projet Latex 2016]{Présentation Projet Latex 2016}
\author{SETI 3e Année}
\institute {\normalsize Cadio Florent,\\
 Le Pichon Yann,\\
Albouy Hugo,\\
Yildirim Herve,\\
Merouane Mehdi,\\
Rafidison Michael,\\}

\date{25 Janvier 2016}
\begin{document}

\begin{frame}
\titlepage
	\begin{columns}[b]
	\begin{column}{2cm}
		\raggedbottom{		
		\begin{flushleft}
			\includegraphics[width=20mm,height=8mm]{CNAM.jpg}
		\end{flushleft}}
	\end{column}
	\begin{column}{2cm}
		\raggedbottom{
		\begin{flushleft}
			\includegraphics[width=20mm,height=8mm]{Scilab_logo.jpg}
		\end{flushleft}}
	\end{column}
	\end{columns}
\end{frame}

%Sommaire

\section{Sommaire}
\begin{frame}[label=sommaire]
\frametitle{Sommaire}
\tableofcontents
\end{frame}

%Introduction

\AtBeginSection[]
{
  \begin{frame}
  \frametitle{Sommaire}
  \tableofcontents[currentsection]
  \end{frame} 
}
\section{Introduction}
\begin{frame}[label=Introduction]

\frametitle{Introduction}
\begin{beamerboxesrounded}[shadow=true]{Partie 1 : Traitement données}
		\begin{itemize}
			\item Présentation Traitement
			\item Affichage Latex PDF
			\item Affichage Latex HTML
		\end{itemize}
\end{beamerboxesrounded}
\pause
\begin{beamerboxesrounded}[shadow=true]{Partie 2 : Estimateur}
	\begin{itemize}
			\setbeamertemplate{itemize item}[ball]
			\item Résolution du système d'équation
			\item Transcodage pour estimer les différents paramètres
			\item Sauvegarde des échantillons dans les nouveaux fichiers .dat
			\item Regénérer partie 1 à partir de ces nouveaux fichiers
		\end{itemize}
\end{beamerboxesrounded}
\end{frame}

%Projet présentation des différentes données

\AtBeginSection[]
{
  \begin{frame}
  \frametitle{Sommaire}
  \tableofcontents[currentsection]
  \end{frame} 
}
\section{Présentation du Projet}

\subsection{ Executable}
\begin{frame}[label=Executable]
\frametitle{Script batch d'éxécution}
\begin{beamerboxesrounded}[shadow=true]{Executable}
		\begin{enumerate}
			\item Lancement Scilab 
			\item Exécution du main
			\item Compilation des fichiers TEX
			\item Affichage des 5 fichiers PDF/HTML
		\end{enumerate}
\end{beamerboxesrounded}
\end{frame}


\subsection{ Main}
\begin{frame}[label=Main]
\frametitle{Fonction Main}
\begin{beamerboxesrounded}[shadow=true]{Main}
		\begin{enumerate}
			\item Le main a une boucle qui gère les 4 types de données
			\item A chaque itération, traitement + affichage TEX	
		\end{enumerate}
\end{beamerboxesrounded}
\end{frame}


\subsection{ Données d'entrées}
\begin{frame}[label=Données]
\frametitle{Données d'entrées}
\begin{beamerboxesrounded}[shadow=true]{Données}
		\begin{enumerate}
			\item Données Continues Groupées (DCG)\\
			\item Données Continues Non Groupées (DCNG)\\
			\item Données Discrètes Groupées (DDG)\\
			\item Données Discrètes Non Groupées (DDNG)
		\end{enumerate}
\end{beamerboxesrounded}
\end{frame}

\subsection{Traitement}
\begin{frame}[label=Traitement]
\frametitle{Traitement}
\begin{beamerboxesrounded}[shadow=true]{Traitement}
		\begin{enumerate}
			\item Traitements des 4 types de données
	
		\end{enumerate}
\end{beamerboxesrounded}
\end{frame}


\subsection{Création fichier Latex}
\begin{frame}[label=Création fichier Latex]
\frametitle{Affichage Latex}
\begin{beamerboxesrounded}[shadow=true]{Latex}
		\begin{enumerate}
			\item Opérations effectuées
			\item Tableau
			\item Histogramme
			\item Génération du PDF/HTML
		\end{enumerate}
\end{beamerboxesrounded}
\end{frame}




% Première Partie
\AtBeginSection[]
{
  \begin{frame}
  \frametitle{Sommaire}
  \tableofcontents[currentsection]
  \end{frame} 
}
\section{Résultats Partie 1}
\begin{frame}[label=Résultats]
\frametitle{Résultats Partie 1}
\begin{beamerboxesrounded}[shadow=true]{Fonction 1 : DCG}
	\begin{itemize}
			\item \small Données Continue Groupées
			\item \small Calcul des paramètres et affichage de l'histogramme
	\end{itemize}
\end{beamerboxesrounded}

\begin{beamerboxesrounded}[shadow=true]{Fonction 2 : DCNG}
	\begin{itemize}
			\item \small Données Continue Non Groupées
			\item \small Calcul des paramètres et affichage de l'histogramme
	\end{itemize}
\end{beamerboxesrounded}

\begin{beamerboxesrounded}[shadow=true]{Fonction 3 : DDG}
	\begin{itemize}
			\item \small Données Discrètes Groupées
			\item \small Calcul des paramètres et affichage de l'histogramme
	\end{itemize}
\end{beamerboxesrounded}

\begin{beamerboxesrounded}[shadow=true]{Fonction 4 : DDNG}
	\begin{itemize}
			\item \small Données Discrète Non Groupées
			\item \small Calcul des paramètres et affichage de l'histogramme
	\end{itemize}
\end{beamerboxesrounded}
\end{frame}

\subsection{ Fonction 1 : DCG}
\begin{frame}[label=Fonction 1 : DCG]
\frametitle{Fonction 1 : DCG}
\begin{beamerboxesrounded}[shadow=true]{Données Continues Groupées}
\includegraphics[scale=0.45]{TableauxDCG}\\
\end{beamerboxesrounded}
\end{frame}


\subsection{Fonction 2 : DCNG}
\begin{frame}[label=Fonction 2 : DCNG]
\frametitle{Fonction 2 : DCNG}
\begin{beamerboxesrounded}[shadow=true]{Données Continues Non Groupées}
	\includegraphics[scale=0.35]{TableauxDCNG}\\
\end{beamerboxesrounded}
\end{frame}


\subsection{Fonction 3 : DDG}
\begin{frame}[label=Fonction 3 : DDG]
\frametitle{Fonction 3 : DDG}
\begin{beamerboxesrounded}[shadow=true]{Données Discrètes Groupées}
			\includegraphics[scale=0.4]{TableauxDDG}\\
\end{beamerboxesrounded}
\end{frame}


\subsection{Fonction 4 : DDNG}
\begin{frame}[label=Fonction 4 : DDNG]
\frametitle{Fonction 4 : DDNG}
\begin{beamerboxesrounded}[shadow=true]{Données Discrètes Non Groupées}
	\includegraphics[scale=0.4]{TableauxDDNG}\\
\end{beamerboxesrounded}
\end{frame}

\AtBeginSection[]
{
  \begin{frame}
  \frametitle{Sommaire}
  \tableofcontents[currentsection]
  \end{frame} 
}

\section{Résultats Partie 2}

\subsection{DCNG mu}
\begin{frame}[label=Fonction 1 : DCNG avec estimation de mu]
\frametitle{Fonction 1 : DCNG avec estimation de mu}
\begin{beamerboxesrounded}[shadow=true]{Données Continues Non Groupées}
\includegraphics[scale=0.25]{TableauxDCNGp2}\\
\end{beamerboxesrounded}
\end{frame}

\subsection{DCNG sigma}
\begin{frame}[label=Fonction 2 : DCNG avec estimation de sigma]
\frametitle{Fonction 2 : DCNG avec estimation de Sigma}
\begin{beamerboxesrounded}[shadow=true]{Données Continues Non Groupées}
\includegraphics[scale=0.35]{TableauxDCNGp2sigma}\\
\end{beamerboxesrounded}
\end{frame}

\AtBeginSection[]
{
  \begin{frame}
  \frametitle{Sommaire}
  \tableofcontents[currentsection]
  \end{frame} 
}
\section{Conclusion}
\begin{frame}[label=Conclusion]

\frametitle{Conclusion}
\begin{beamerboxesrounded}[shadow=true]{Conclusion}
		\begin{itemize}
			\item Conclusion générale
		\end{itemize}
\end{beamerboxesrounded}
\end{frame}

\end{document}
