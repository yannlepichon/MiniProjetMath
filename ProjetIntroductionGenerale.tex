\documentclass[a4paper,12pt,reqno]{article}
\usepackage{amsmath}
\usepackage{amsfonts}
\usepackage{amssymb}
\usepackage{graphicx}
\newcommand{\newparagraphe}[1]{\paragraph{#1}\mbox{}\\}
\begin{document}
\begin{center}
{ \huge Projet Latex/Scilab} 
\newparagraphe{}
\includegraphics[scale=0.5]{Scilab_logo}
\\
\\
\includegraphics[scale=0.5]{LaTeX_logo}
\newparagraphe{}
AUTEURS :
\newparagraphe{}
 Cadio Florent , Le Pichon Yann, Albouy Hugo,ouatik said\\ Yildirim Herve, Merouane Mehdi, Rafidison Michael 
\newparagraphe{}
\end{center}
\newparagraphe{}
\textbf{R\'esum\'e} Ce projet nous permettra de prendre en main Scilab et de g\'en\'erer une documentation propre avec le logiciel Latex via le traitement de fichiers de donn\'ees
\pagebreak 
\tableofcontents 
\cleardoublepage
\pagestyle{plain}
\newparagraphe{}
\section{Introduction}
Ceci est une introduction g\'en\'erale au projet de mod\'elisation statistique, elle pr\'esentera les diff\'erentes donn\'ees utilis\'ees et le travail qui a \'et\'e effectu\'e. Ce document fait suite \`{a} une demande r\'ealis\'ee par monsieur Dariush Ghorbanzadeh, dans le cadre du projet de derni\`{e}re ann\'ee du cycle ing\'enieur. 
\newparagraphe{}
La pr\'esente demande, est de d\'evelopper et de r\'ealiser un outil de mod\'elisation statistique \`{a} l\'\ aide du logiciel Scilab. A partir de fichiers contenant des donn\'ees, nous devons effectuer diff\'erents traitements statistiques,  estimations et mod\'elisations et les afficher \`{a} l\'\ aide de Latex en format pdf et html.
\newparagraphe{}
Lors de ce projet, nous aurons \`{a} mettre en pratique  les notions th\'eoriques acquises lors de ces deux derni\`{e}res ann\'ees.
\pagebreak 

\section{Organisation du travail}
\newparagraphe{}
\includegraphics[scale=0.8]{MSprojectv1}\\
\pagebreak 

\section{Pr\'esentation du projet}
\newparagraphe{}

Le projet est divis\'e en deux grandes parties distinctes, la premi\`{e}re permet de traiter quatre types de donn\'ees et de calculer diff\'erentes informations les concernant, et la seconde consiste \`{a} l\'\ utilisation de l\'\ Estimateur Maximum de Vraissemblance, ce qui nous permettra de remplacer les donn\'ees qui figurent sur les fichiers .Dat par des donn\'ees estim\'ees.

\subsection{Partie 1 : Traitement des donnees }
\newparagraphe{}
Nous allons par la suite traiter quatre types de donn\'ees : donn\'ees continues/discr\`{e}tes group\'ees, et donn\'ees continues/discr\`{e}tes non group\'ees. 
Tout d\'\ abord d\'efinissons ce qu\'\ est une donn\'ee continue, et ce qu\'\ est une donn\'ee discr\`{e}te. Puis nous expliquerons la diff\'erence entre des donn\'ees group\'ees, et des non group\'ees.
\\
\\

\textbf{Donn\'ees continues} :
\\Une variable continue peut prendre, en th\'eorie, une infinit\'e de valeurs, formant un ensemble continu.\\

\textbf{Donn\'ees discr\`{e}tes} :
\\Une variable discr\`{e}te \`{a} une valeur finie. Il est possible de les \'enum\'erer (  1, 2, 3,Š).\\ 

\textbf{Donn\'ees group\'ees} :
\\Une donn\'ee est dite group\'ee lorsqu elle a \'et\'e tri\'ee en fonction d\'\ autres param\`{e}tres  par exemple le nombre de fois que celle-ci est utilis\'ee (effectif).\\

\textbf{Donn\'ees non group\'ees} :
\\Une donn\'ee est dite non group\'ee lorsqu\'\ elle n\'\ a pas \'et\'e tri\'ee en fonction d\'\ un autre param\`{e}tre, elle sera exprim\'ee sur une seule colonne dans un tableau de donn\'ee.\\


\subsubsection{moyenne g\'eom\'etrique}
\includegraphics[scale=0.5]{moyennegeo}\\

\subsubsection{moyenne quadratique}
\includegraphics[scale=0.5]{moyennequadratique}\\

\subsubsection{moyenne arithm\'etique}
\includegraphics[scale=0.5]{moyennearithmetique}\\

\subsubsection{moyenne harmonique}
\includegraphics[scale=0.5]{moyenneharmonique}\\

\subsubsection{moment d\'\ ordre n}
\includegraphics[scale=0.5]{momentdordre}\\

\subsubsection{moment centr\'e d\'\ ordre n}
\includegraphics[scale=0.5]{momentcentredordre}\\

\newparagraphe{}
\pagebreak

\subsection{Partie 2 : estimateur }
\newparagraphe{}

Cette seconde partie est d\'ecompos\'ee en deux sous parties, la th\'eorie puis la simulation.\\
L\'\ objectif de cette seconde partie est de r\'esoudre un syst\`{e}me d\'\ \'equation qui permettra de fournir les estimateurs.\\
Puis nous transformons nos \'equations en code Scilab pour estimer les param\`{e}tres mu, c, sigma et k.\\
Puis nous sauvegardons les \'echantillons des param\`{e}tres dans un fichier .dat.\\
Et pour finir nous utilisons ces nouvelles donn\'ees calcul\'ees \`{a} la place des fichiers .dat initiaux.\\

\subsubsection{Partie Th\'eorique : R\'esolution de l\'\ \'equation }\\
\\
\\
\\
\textbf{ Vraisemblance associ\'ee} :\\
\includegraphics[scale=0.75]{vraisemblanceassocie}\\\\

\textbf{ Log Vraisemblance associ\'ee au mod\`{e}le} :\\
\includegraphics[scale=0.75]{logvraisemblanceassocie}\\
\\
\pagebreak

\textbf{ Estimateurs du maximum de vraisemblance - D\'eriv\'ees} :
\\
\\
$ \frac{\delta}{\delta\mu} l(X,\Theta,k) = \frac{1}{\sigma^2} \sum_{i = 1}^k X_i - \mu + \frac{1}{c^2\sigma^2} \sum_{i = k+1}^n (X_i - \mu)\\


$ \frac{\delta}{\delta\sigma^2} l(X,\Theta,k) = \frac{-n}{2\sigma^2} + \frac{1}{2(\sigma^2)^2} \sum_{i = 1}^k (X_i - \mu)^2
$ \frac{1}{2 c^2(\sigma^2)^2}  
$ \sum_{i = k+1}^n (X_i - \mu)^2
\\

$\frac{\delta}{\delta c^2} l(X,\Theta,k) = -\frac{n-k}{2} \frac{1}{c^2} + \frac{1}{2 (c^2)^2\sigma^2} \sum_{i = k+1}^n (X_i - \mu)^2
\\
\\
\\
\\
\textbf{ Estimateurs du maximum de vraisemblance - estimation des param\`{e}tres} :
\\
\\
$ \mu = \frac{S_n + \frac{1}{c^2} S_{n-k}}{k + \frac{1}{c^2} (n-k)}
\\
$ c^2 = \frac{\mu(n-k)-S_{n-k}}{S_k - \mu k}
\\
\sigma^2 = \frac{\sum{X_i^2} + \mu^2(n-k) - 2\mu \sum{X_i}}{c^2(n-k)}


\pagebreak



\section{R\'esultat obtenus}
\subsection{Partie 1 : }
\subsubsection{DCG}
\includegraphics[scale=0.75]{TableauxDCG}\\
\subsubsection{DCNG}
\includegraphics[scale=0.75]{TableauxDCNG}\\
\subsubsection{DDG}
\includegraphics[scale=0.75]{TableauxDDG}\\
\subsubsection{DDNG}
\includegraphics[scale=0.75]{TableauxDDNG}\\
\pagebreak

\subsection{Partie 2 : }
\subsubsection{Donn\'ees Continues Non Group\'ees avec Estimation de mu}
\includegraphics[scale=0.5]{TableauxDCNGp2}\\
\subsubsection{Donn\'ees Continues Non Group\'ees avec Estimation de sigma}}
\includegraphics[scale=0.5]{TableauxDCNGp2sigma}\\


\end{document}
