\documentclass[a4paper,12pt,reqno]{article}
\usepackage{amsmath}
\usepackage{amsfonts}
\usepackage{amssymb}
\usepackage{graphicx}
\newcommand{\newparagraphe}[1]{\paragraph{#1}\mbox{}\\}
\begin{document}
\begin{center}
{ \huge Projet Latex/Scilab} 
\newparagraphe{}
\includegraphics[scale=0.5]{Scilab_logo}
\newparagraphe{}
AUTEURS :
\newparagraphe{}
 Cadio Florent , Le Pichon Yann, Albouy Hugo,Ouatik Said\\ Yildirim Herve, Merouane Mehdi, Rafidison michael 
\newparagraphe{}
\newparagraphe{}
\newparagraphe{}
{ A l intention de monsieur Ghorbanzadeh} 
\end{center}
\newparagraphe{}
\begin{Resume}
Resume Ce projet nous permettra de prendre en main Scilab et de generer une documentation propre avec le logiciel Latex via le traitement de fichiers de donnees
\end{Resume}
\pagebreak 
\tableofcontents 
\cleardoublepage
\pagestyle{plain}
\newparagraphe{}
\section{Introduction}
Ceci est une introduction generale au projet de modelisation statistique, elle presentera les differentes donnees utilisees et le travail qui a ete effectue. Ce document fait suite a une demande realisee par monsieur Dariush Ghorbanzadeh, dans le cadre du projet de derniere annee du cycle ingenieur. 
\newparagraphe{}
La presente demande, est de developper et de realiser un outil de modélisation statistique a l aide du logiciel Scilab. A partir de fichiers contenant des donnees, nous devons effectuer differents traitements statistiques,  estimations et modelisations et les afficher a l aide de Latex en format pdf et html.
\newparagraphe{}
Lors de ce projet, nous aurons a mettre en pratique  les notions theoriques acquises lors de ces deux dernieres annees.
\pagebreak 

\section{Organisation du travail}
\newparagraphe{}

inserer ici le msproject + le travail de chacun
\pagebreak 

\section{Presentation du projet}
\newparagraphe{}

Le projet est divise en deux grandes parties distinctes, dans la premiere il s agit de traiter quatre types de donnees et de calculer différentes informations les concernant, et dans la seconde il s agit de mettre en place un estimateur BLABLABLA.

\subsection{Partie 1 : }
\newparagraphe{}
Nous allons par la suite traiter quatre types de donnees : donnees continues/discretes groupees, et donnees continues/discretes non groupees. 
Tout d abord definissons ce qu est une donnee continue, et ce qu est une donnee discrete. Puis nous expliquerons la difference entre des donnees groupees, et des non groupees.
\\
\\

\textbf{Donnees continues} :
\\Une variable continue peut prendre, en theorie, une infinite des valeurs, formant un ensemble continu.\\

\textbf{Donnees discretes} :
\\Une variable discrete a une valeur finie. Il est possible de les enumerer (  1, 2, 3,… ).\\ 

\textbf{Donnees groupees} :
\\Une donnee est dite groupee lorsqu elle a été triee en fonction d autres parametres  par exemple le nombre de fois que celle-ci est utilise (effectif).\\

\textbf{Donnees non groupees} :
\\Une donnee est dite non groupee lorsqu elle n a pas ete triee en fonction d un autre parametre, elle sera exprimee sur une seule colonne dans un tableau de donnee.\\


\subsubsection{moyenne geometrique}
\includegraphics[scale=0.5]{moyennegeo}\\

\subsubsection{moyenne quadratique}
\includegraphics[scale=0.5]{moyennequadratique}\\

\subsubsection{moyenne arithmetique}
\includegraphics[scale=0.5]{moyennearithmetique}\\

\subsubsection{moyenne harmonique}
\includegraphics[scale=0.5]{moyenneharmonique}\\

\subsubsection{moment d ordre n}
\includegraphics[scale=0.5]{momentdordre}\\

\subsubsection{moment centre d ordre n}
\includegraphics[scale=0.5]{momentcentredordre}\\

\newparagraphe{}
\subsection{Resultats obtenus}


\subsection{Histogrammes}

\end{document}
